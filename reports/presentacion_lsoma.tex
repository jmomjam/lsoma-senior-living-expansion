% =============================================================================
% PRESENTACIÓN L-SOMA - Algoritmo de Localización de Residencias
% =============================================================================
\documentclass[aspectratio=169,11pt]{beamer}

% Tema y colores
\usetheme{Madrid}
\usecolortheme{whale}
\setbeamertemplate{navigation symbols}{}
\setbeamertemplate{footline}[frame number]

% Paquetes
\usepackage[utf8]{inputenc}
\usepackage[spanish]{babel}
\usepackage{graphicx}
\usepackage{booktabs}
\usepackage{amsmath}
\usepackage{eurosym}
\usepackage{tikz}

% Configuración de gráficos
\graphicspath{{./}}

% Colores personalizados
\definecolor{lsomaBlue}{RGB}{41, 128, 185}
\definecolor{lsomaGreen}{RGB}{39, 174, 96}
\definecolor{lsomaOrange}{RGB}{230, 126, 34}
\definecolor{lsomaRed}{RGB}{231, 76, 60}

\setbeamercolor{title}{fg=white,bg=lsomaBlue}
\setbeamercolor{frametitle}{fg=white,bg=lsomaBlue}

% Información del documento
\title[L-SOMA]{\textbf{L-SOMA}}
\subtitle{Location-based System for Optimal Market Allocation\\[0.3cm]
\small Identificación de Ubicaciones Óptimas para Residencias de Mayores}
\author{Johan Manuel Orozco Mesa}
\date{Enero 2026}
\institute{Caso Práctico de IA}

% =============================================================================
\begin{document}
% =============================================================================

% -----------------------------------------------------------------------------
% PORTADA
% -----------------------------------------------------------------------------
\begin{frame}
\titlepage
\end{frame}

% -----------------------------------------------------------------------------
% ÍNDICE
% -----------------------------------------------------------------------------
\begin{frame}{Contenido}
\tableofcontents
\end{frame}

% =============================================================================
\section{Contexto y Objetivo}
% =============================================================================

\begin{frame}{El Reto Estratégico}
\begin{columns}
\column{0.6\textwidth}
\textbf{Objetivo del proyecto:}
\begin{itemize}
    \item Evaluar viabilidad de \textbf{1.000 nuevas residencias}
    \item \textbf{100 camas} por residencia
    \item Precio plaza: \textbf{2.100\euro{}/mes}
    \item Ocupación objetivo: \textbf{85\%}
\end{itemize}

\vspace{0.5cm}
\textbf{Basado en:}\\
\textit{``Disparadores de Ingreso a Residencias Privadas''}

\column{0.4\textwidth}
\begin{block}{Pregunta clave}
¿Dónde construir para maximizar ocupación y minimizar riesgo?
\end{block}
\end{columns}
\end{frame}

% -----------------------------------------------------------------------------
\begin{frame}{Arquitectura del Algoritmo L-SOMA}
\centering
\includegraphics[width=0.85\textwidth]{arquitectura_lsoma.pdf}
\end{frame}

% =============================================================================
\section{Metodología}
% =============================================================================

\begin{frame}{Pipeline de Análisis}
\begin{enumerate}
    \item \textbf{Definición del Target:} Vector Q (mujeres 80+, renta >30k\euro{})
    \item \textbf{Matriz Demográfica P:} 32.910 secciones $\times$ 42 columnas
    \item \textbf{Resonancia (JSD):} Divergencia Jensen-Shannon
    \item \textbf{Clustering DBSCAN:} $\varepsilon$=1.5km geodésico
    \item \textbf{Masa Crítica:} Umbral $\geq$85 camas
    \item \textbf{Validación Competencia:} Google Places API
\end{enumerate}

\vspace{0.5cm}
\begin{center}
\textbf{32.910 secciones} $\rightarrow$ \textbf{135 clusters} $\rightarrow$ \textbf{21 viables (Prime)}
\end{center}
\end{frame}

% -----------------------------------------------------------------------------
\begin{frame}{Divergencia Jensen-Shannon: Resonancia Demográfica}
\begin{columns}
\column{0.5\textwidth}
\textbf{Fórmula:}
\[
\mathcal{S}_i = 1 - JSD(\vec{P}_i \| \vec{Q})
\]

\textbf{Interpretación:}
\begin{itemize}
    \item $\mathcal{S}_i \approx 1$: Alta resonancia
    \item $\mathcal{S}_i \approx 0$: Baja resonancia
\end{itemize}

\column{0.5\textwidth}
\includegraphics[width=\textwidth]{histograma_resonancia.pdf}
\end{columns}
\end{frame}

% -----------------------------------------------------------------------------
\begin{frame}{Matriz Demográfica P}
\centering
\includegraphics[width=0.75\textwidth]{matriz_p_heatmap.pdf}

\vspace{0.3cm}
\small 32.910 secciones $\times$ 42 columnas demográficas
\end{frame}

% =============================================================================
\section{Resultados}
% =============================================================================

\begin{frame}{Clustering Espacial: Mapa de España}
\centering
\includegraphics[width=0.8\textwidth]{mapa_clusters_espana.pdf}
\end{frame}

% -----------------------------------------------------------------------------
\begin{frame}{Top 10 Clusters por Potencia}
\centering
\includegraphics[width=0.85\textwidth]{barras_top10_clusters.pdf}
\end{frame}

% -----------------------------------------------------------------------------
\begin{frame}{Escenarios de Viabilidad}
\begin{table}
\centering
\small
\begin{tabular}{lccc}
\toprule
\textbf{Escenario (Cuota)} & \textbf{Clusters} & \textbf{Residencias} & \textbf{Gap} \\
\midrule
1,5\% (Muy Conservador) & 6 & 17 & -983 \\
2,0\% (Conservador) & 12 & 29 & -971 \\
3,0\% (Base) & 19 & 50 & -950 \\
5,0\% (Agresivo) & 25 & 91 & -909 \\
\bottomrule
\end{tabular}
\end{table}

\vspace{0.5cm}
\begin{alertblock}{Conclusión}
\textbf{Gap estructural:} El mercado español soporta máximo 64-359 residencias orgánicamente. Para 1.000 se requiere M\&A.
\end{alertblock}
\end{frame}

% -----------------------------------------------------------------------------
\begin{frame}{Análisis de Sensibilidad: Market Share}
\centering
\includegraphics[width=0.8\textwidth]{sensibilidad_share.pdf}
\end{frame}

% =============================================================================
\section{Expansión Adaptativa}
% =============================================================================

\begin{frame}{Algoritmo de Expansión}
\textbf{Objetivo:} Maximizar residencias relajando parámetros de forma controlada

\vspace{0.3cm}
\begin{table}
\centering
\small
\begin{tabular}{lccc}
\toprule
\textbf{Parámetro} & \textbf{Prime} & \textbf{Expandido} & \textbf{Riesgo} \\
\midrule
Percentil Score & 85 & 60 & Bajo \\
Market Share & 3\% & 6\% & Medio \\
Penalización Renta & 0,40 & 0,70 & Alto \\
Camas Mínimas & 85 & 60 & Crítico \\
\bottomrule
\end{tabular}
\end{table}

\vspace{0.3cm}
\textbf{Resultado:} 64 $\rightarrow$ \textbf{359 residencias} (72 iteraciones)
\end{frame}

% -----------------------------------------------------------------------------
\begin{frame}{Convergencia del Algoritmo}
\centering
\includegraphics[width=0.85\textwidth]{expansion_convergencia.pdf}
\end{frame}

% -----------------------------------------------------------------------------
\begin{frame}{Comparativa Prime vs Expandido}
\centering
\includegraphics[width=0.85\textwidth]{expansion_comparativa.pdf}
\end{frame}

% -----------------------------------------------------------------------------
\begin{frame}{Mapa: Clusters Viables tras Expansión}
\centering
\includegraphics[width=0.85\textwidth]{expansion_mapa_clusters.pdf}
\end{frame}

% =============================================================================
\section{Validación de Competencia}
% =============================================================================

\begin{frame}{Análisis de Competencia: Google Places API}
\textbf{Metodología:}
\begin{itemize}
    \item Radio de búsqueda: 1.500m
    \item Query: ``Residencia de ancianos OR Geriátrico''
    \item Camas estimadas/competidor: 75 (media IMSERSO)
\end{itemize}

\vspace{0.3cm}
\textbf{Índice de Saturación:}
\[
I_{sat} = \frac{N_{competidores} \times 75}{Camas_{potenciales}}
\]

\vspace{0.3cm}
\textbf{Clasificación:}
\begin{itemize}
    \item \textcolor{lsomaBlue}{\textbf{Blue Ocean:}} $I_{sat} < 0.20$
    \item \textcolor{lsomaOrange}{\textbf{Batalla:}} $0.20 \leq I_{sat} \leq 1.0$
    \item \textcolor{lsomaRed}{\textbf{Saturado:}} $I_{sat} > 1.0$
\end{itemize}
\end{frame}

% -----------------------------------------------------------------------------
\begin{frame}{Distribución por Nivel de Competencia}
\begin{columns}
\column{0.55\textwidth}
\centering
\includegraphics[width=\textwidth]{competencia_distribucion.pdf}

\column{0.45\textwidth}
\begin{table}
\centering
\small
\begin{tabular}{lcc}
\toprule
\textbf{Tipo} & \textbf{N} & \textbf{\%} \\
\midrule
\textcolor{lsomaBlue}{Blue Ocean} & 3 & 3,3\% \\
\textcolor{lsomaOrange}{Batalla} & 13 & 14,3\% \\
\textcolor{lsomaRed}{Saturado} & 75 & 82,4\% \\
\midrule
\textbf{Total} & 91 & 100\% \\
\bottomrule
\end{tabular}
\end{table}
\end{columns}
\end{frame}

% -----------------------------------------------------------------------------
\begin{frame}{Los 3 Blue Oceans Identificados}
\begin{table}
\centering
\begin{tabular}{clccc}
\toprule
\textbf{Rank} & \textbf{Ubicación} & \textbf{Camas} & \textbf{Compet.} & \textbf{$I_{sat}$} \\
\midrule
1 & \textbf{Madrid Centro} & 7.274 & 7 & 0,077 \\
2 & \textbf{Bilbao} & 2.114 & 4 & 0,151 \\
3 & \textbf{Málaga Costa} & 419 & 1 & 0,191 \\
\bottomrule
\end{tabular}
\end{table}

\vspace{0.5cm}
\begin{block}{Nota Estratégica}
\textbf{Madrid y Bilbao:} Alta barrera inmobiliaria. Oportunidad en reconversión de edificios o modelo High-End.\\
\textbf{Málaga:} Menor barrera. Viable para modelo greenfield tradicional.
\end{block}
\end{frame}

% =============================================================================
\section{Conclusiones}
% =============================================================================

\begin{frame}{Hallazgos Principales}
\begin{enumerate}
    \item \textbf{Capacidad máxima:} 64-348 residencias (construcción orgánica)
    
    \item \textbf{Concentración metropolitana:} Top 5 clusters = 56\% de la potencia\\
    (Madrid, Barcelona, Bilbao, Zaragoza, Valladolid)
    
    \item \textbf{Objetivo 1.000 inalcanzable:} Gap de 641 unidades $\rightarrow$ requiere M\&A
    
    \item \textbf{Punto óptimo:} Iteración 40 (348 residencias, calidad preservada)
    
    \item \textbf{82,4\% saturado:} Solo 3 clusters califican como Blue Ocean
    
    \item \textbf{Madrid Centro:} Mayor oportunidad ($I_{sat}$=0,077)
\end{enumerate}
\end{frame}

% -----------------------------------------------------------------------------
\begin{frame}{Recomendación Estratégica}
\begin{columns}
\column{0.5\textwidth}
\textbf{Construcción Orgánica:}
\begin{itemize}
    \item 64-100 residencias Prime
    \item Foco: Madrid, Barcelona, Bilbao
    \item Modelo greenfield: Málaga
\end{itemize}

\vspace{0.3cm}
\textbf{Expansión Moderada:}
\begin{itemize}
    \item Hasta 250 residencias adicionales
    \item Relajar percentil/share
    \item Preservar camas $\geq$85
\end{itemize}

\column{0.5\textwidth}
\textbf{M\&A para el Gap:}
\begin{itemize}
    \item 641 residencias restantes
    \item Adquisición de operadores
    \item Fuera del alcance L-SOMA
\end{itemize}

\vspace{0.5cm}
\begin{alertblock}{Total estimado}
\textbf{350-400 residencias} mediante construcción orgánica + expansión
\end{alertblock}
\end{columns}
\end{frame}

% -----------------------------------------------------------------------------
\begin{frame}{Limitaciones del Modelo}
\begin{itemize}
    \item \textbf{Horizonte estático:} Snapshot del censo 2021-2025. No proyecta envejecimiento futuro.
    
    \item \textbf{Incertidumbre en competencia:} API Google no captura 100\% de oferta. 75 camas/competidor es aproximación.
    
    \item \textbf{Proxy de burnout:} Asume mujeres 45-64 años son cuidadoras locales (ignora movilidad).
    
    \item \textbf{Market share hipotético:} 1,5-6\% sin validación empírica.
\end{itemize}

\vspace{0.5cm}
\begin{block}{Siguiente paso}
Validación de campo en los 3 Blue Oceans identificados
\end{block}
\end{frame}

% -----------------------------------------------------------------------------
\begin{frame}
\centering
\vspace{2cm}
{\Huge \textbf{¿Preguntas?}}

\vspace{1cm}
\textbf{L-SOMA}\\
Location-based System for Optimal Market Allocation

\vspace{1cm}
{\small Johan Manuel Orozco Mesa | Enero 2026}
\end{frame}

% =============================================================================
\end{document}
